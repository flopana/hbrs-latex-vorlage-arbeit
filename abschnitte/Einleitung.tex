\chapter{Der Weg zur Praxisprojektstelle}\label{ch:weg_zur_pp}
Im August 2024 habe ich mich bei der lise GmbH als Werkstudent in der Softwareentwicklung beworben, unter der Voraussetzung, dass ich dort  
mein Praxisprojekt absolvieren kann, da dies in naher Zukunft auf mich zukommen würde.\\  
Von Vorteil war hier meine vorherige Erfahrung als Softwareentwickler bei meinen vorherigen Arbeitgebern, wodurch wir ein  
Thema finden konnten, das der lise GmbH sowohl Mehrwert bringt als auch in den Rahmen des Praxisprojekts passt.\\  
Das war nicht ganz leicht, da der übliche Projektalltag von den Vorgaben der Hochschule, ein klar abgegrenztes Projekt zu behandeln, abweicht.  
Trotzdem konnte ein passendes Projekt gefunden werden, mit dem alle Parteien zufrieden sind.

\chapter{Unternehmensbeschreibung}\label{ch:unternehmen}
\section{Unternehmensdaten}\label{sec:nternehmensdaten}
lise GmbH

\textbf{Hauptsitz:} Rudi-Conin-Straße 5, 50829 Köln\\
\textbf{Gründungsjahr:} 1999\\
\textbf{Mitarbeiter: } 100\\
\textbf{Registernummer: } HRB 61293\\
\textbf{Bilanzsumme 2023: } 4.907.296,24 EUR \autocite{unternehmensregister2024}\\
\textbf{Website:} \url{https://www.lise.de/}\\
\textbf{Branchenzugehörigkeit:} Softwareentwicklungs- und IT-Consultings-Dienstleistungen\\
\textbf{Logo:}
\begin{figure}[H]
    \begin{center}
        \includegraphics[width=3cm]{bilder/lise_logo_web.png}
        \caption{Logo lise GmbH}\label{fig:lise}
    \end{center}
\end{figure}

\section{Dienstleistungen}\label{sec:dienstleistungen}
Die lise GmbH bietet maßgeschneiderte Softwarelösungen für die individuellen Ansprüche ihrer Kunden.\\
Diese erstrecken sich neben der Softwareentwicklung auch über Künstliche Intelligenz, UX-Design, Scrum,
Softwaremodernisierung, Sharepoint DevOps und generelle Beratung bezüglich IT-Strategie.\autocite[/ueber-uns/unternehmen]{LiseWeb}

Die Kunden der lise sind aus diversen Industrien z.B Großhandel, Lebensmittel, Behörden und natürlich auch der IT-Branche.




% \section{Zitate}\label{sec:zitate}
% Nach~\cite{Bartels2008} wird ein Beispieltext vor dem ersten Zitat in einem Kapitel benötigt.\\
% \textcite{Yoshida1999} sagen hier steht ein Text. \\
% \textcite*{Burns2003} sagen hier steht ein Text. \\
% \enquote{Hier steht ein Text.}\autocite{Bartels2008} \\
% \enquote{Hier steht ein Text.}\autocite[Vgl.][]{Bartels2008} \\
% \enquote{Hier steht ein Text.}\autocite[][S. 200]{Bruno2009} \\
% \enquote{Hier steht ein Text.}\autocite*[][S. 200]{Bruno2009} \\
% \enquote{Hier steht ein Text.}\autocite{Wikipedia2011, Bruno2009}

% \subsection{Abkürzungen}\label{subsec:abkuerzungen}
% \gls{but}\\
% \gls{cdma}\\
% \gls{gsm}\\
% \gls{ic}\\
% \gls{lh2}\\
% \gls{lox}\\
% \gls{na}\\
% \gls{nad+}\\
% \gls{nua}\\
% \gls{tdma}\\
% \gls{ua}

% \subsection{Quellcode}\label{subsec:quellcode}
% \begin{figure}[H]
%     \begin{center}
%         \begin{minted}{c}
% int main() {
%   printf("hello, world");
%   return 0;
% }
%         \end{minted}
%         \caption{Hier steht eingebetteter Quellcode}\label{fig:hier-steht-eingebetteter-quellcode}
%     \end{center}
% \end{figure}

% \begin{figure}[H]
%     \begin{center}
%         \inputminted{python}{anhang/example.py}
%         \caption{Dieser Quellcode ist in einer externen Datei ausgelagert}\label{fig:dieser-quellcode-ist-in-einer-externen-datei-ausgelagert}
%     \end{center}
% \end{figure}
